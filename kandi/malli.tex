% --- Template for thesis / report with tktltiki2 class ---
% 
% last updated 2013/02/15 for tkltiki2 v1.02

\documentclass[finnish]{tktltiki2}

% tktltiki2 automatically loads babel, so you can simply
% give the language parameter (e.g. finnish, swedish, english, british) as
% a parameter for the class: \documentclass[finnish]{tktltiki2}.
% The information on title and abstract is generated automatically depending on
% the language, see below if you need to change any of these manually.
% 
% Class options:
% - grading                 -- Print labels for grading information on the front page.
% - disablelastpagecounter  -- Disables the automatic generation of page number information
%                              in the abstract. See also \numberofpagesinformation{} command below.
%
% The class also respects the following options of article class:
%   10pt, 11pt, 12pt, final, draft, oneside, twoside,
%   openright, openany, onecolumn, twocolumn, leqno, fleqn
%
% The default font size is 11pt. The paper size used is A4, other sizes are not supported.
%
% rubber: module pdftex

% --- General packages ---

\usepackage[utf8]{inputenc}
\usepackage[T1]{fontenc}
\usepackage{lmodern}
\usepackage{microtype}
\usepackage{amsfonts,amsmath,amssymb,amsthm,booktabs,color,enumitem,graphicx}
\usepackage[pdftex,hidelinks]{hyperref}

%\usepackage[sort]{natbib} uncomment for a harvard-style
\usepackage[square,sort,comma,numbers]{natbib}


% Automatically set the PDF metadata fields
\makeatletter
\AtBeginDocument{\hypersetup{pdftitle = {\@title}, pdfauthor = {\@author}}}
\makeatother

% --- Language-related settings ---
%
% these should be modified according to your language

% babelbib for non-english bibliography using bibtex
\usepackage[fixlanguage]{babelbib}
\selectbiblanguage{finnish}

% add bibliography to the table of contents
\usepackage[nottoc]{tocbibind}
% tocbibind renames the bibliography, use the following to change it back
\settocbibname{Lähteet}

% --- Theorem environment definitions ---

\newtheorem{lau}{Lause}
\newtheorem{lem}[lau]{Lemma}
\newtheorem{kor}[lau]{Korollaari}

\theoremstyle{definition}
\newtheorem{maar}[lau]{Määritelmä}
\newtheorem{ong}{Ongelma}
\newtheorem{alg}[lau]{Algoritmi}
\newtheorem{esim}[lau]{Esimerkki}

\theoremstyle{remark}
\newtheorem*{huom}{Huomautus}


% --- tktltiki2 options ---
%
% The following commands define the information used to generate title and
% abstract pages. The following entries should be always specified:

\title{Ohjelmistojen testattavuuden parantaminen arkkitehtuurin avulla}
\author{Kristian Wahlroos}
\date{\today}
\level{Kandidaatintutkielma}

\abstract{tiivistelmä (100-200 sanaa; kenelle, miksi, millaisessa ympäristössä; tutkimuskysymys; tulokset; impakti)}

% The following can be used to specify keywords and classification of the paper:

\keywords{avainsana 1, avainsana 2, avainsana 3}

% classification according to ACM Computing Classification System (http://www.acm.org/about/class/)
% This is probably mostly relevant for computer scientists
% uncomment the following; contents of \classification will be printed under the abstract with a title
% "ACM Computing Classification System (CCS):"
% \classification{}

% If the automatic page number counting is not working as desired in your case,
% uncomment the following to manually set the number of pages displayed in the abstract page:
%
%\numberofpagesinformation{16 sivua + 10 sivua liitteissä}
%
% If you are not a computer scientist, you will want to uncomment the following by hand and specify
% your department, faculty and subject by hand:
%
% \faculty{Matemaattis-luonnontieteellinen}
% \department{Tietojenkäsittelytieteen laitos}
% \subject{Tietojenkäsittelytiede}
%
% If you are not from the University of Helsinki, then you will most likely want to set these also:
%
% \university{Helsingin Yliopisto}
% \universitylong{HELSINGIN YLIOPISTO --- HELSINGFORS UNIVERSITET --- UNIVERSITY OF HELSINKI} % displayed on the top of the abstract page
% \city{Helsinki}
%


\begin{document}

% --- Front matter ---

\frontmatter      % roman page numbering for front matter

\maketitle        % title page
\makeabstract     % abstract page

\tableofcontents  % table of contents

% --- Main matter ---

\mainmatter       % clear page, start arabic page numbering

\section{Johdanto}

% Write some science here.

Esimerkkilause ja lähdeviite~\cite{esimerkki} ja \cite{esimerkki2}.
jäsennelty asianmukaisesti (kenelle, miksi, millaisessa ympäristössä; ratkaisun lähestymistapa; tutkimuskysymys, tulokset ja impakti; loppulukujen roolitus).



\section{Body text}
purkaa auki teknisemmälle yleisölle samalla käsitteitä määritellen tutkimuskysymyksen, ratkaisuille ympäristöstä esiintulevat vaatimukset ja keskeiset käsitteet näiden asioiden esittelemiseksi.

antaa riittävästi termistöä ja kysymyksenasetteluja, jotta myöhempänä teksissä on mahdollisuus analyysiin ja evaluointiin.
Keskimmäiset luvut keskittyvät kukin "opettamaan" lukijalle yhden osakysymyksen tai näkökulman käsiteltävään asiaan. Lukujen työnjaon tulisi olla selkeä ja niillä tulee olla selkeä oma rakenteensa, jonka valintaperuste on lukijalle kerrottu.

Viimeistä edellinen luku on varsinaisen kontribuution koti: vertailu, soveltaminen, osien synteesi ja mahdolliset evaluointimenetelmät ja -tulokset pääroolissa.

\section{Ohjelmistoarkkitehtuuri}
Jeejee. Paltu.

\subsection{Ohjelmistojen arkkitehtuuri käsitteenä}
Arkkitehtuuri ohjelmistoissa on käsite, josta moni kehittäjä on tietoinen mutta joka ei ole kovin yksiselitteinen eikä siitä ole yksimielisiä määritelmää \cite{solms_what_2012}. Arkkitehtuuri voidaan kuitenkin nähdä karkeasti neljästä eri näkökulmasta  \citep[s. 2-7]{gorton_understanding_2011}:  ohjelmiston rakenteen kuvaajana, kuvaajana ohjelmiston komponenttien välisille suhteille, mallina joka ottaa huomioon ei-toiminnalliset (non-functional) vaatimukset ja yleisenä abstraktiona. 

% kappaleessa selitetään myös riippuvuuksien vähentämisestä
Rakenteen kuvaajana arkkitehtuuri määrittelee ohjelmistojärjestelmän sisäistä rakennetta, joka koostuu useista komponenteista sekä moduuleista. Erilaiset moduulit tekevät tiettyä työtä ja täten erilaiset vastuut on jaettu ohjelmistojärjestelmän sisällä loogiisiin kokoelmiin ja näistä syntyvät kokoelmat tarjoavat halutun toiminnallisuuden.
% kappaleessa myös patterneja
Komponenttien välisen kommunikaation mallintaminen tulee vastaan, kun ohjelmistojärjestelmää jaetaan erilaisiin komponentteihin. Tavoitteena on kertoa mallintamalla, että mitkä komponentit tai moduulit ovat vuorovaikutuksessa toistensa kanssa ja millä tavoin. Yleisin kommunikaatiotapa on esimerkiksi suorat funktiokutsut komponenttien välillä.  
% kappaleessa ei-toiminnallisten vaatimuksien jako kolmeen eri luokkaan
Ei-toiminnaliset vaatimukset tulevat esille vasta arkkitehtuurin avulla ja arkkitehtuurin mallinnuksen avulla voidaan määrittää miten ohjelma suorittaa sille määrättyä tehtävää sen sijasta, että mallinnettaisiin mitä ohjelma tekee. 
% lähteitä lisää?
Arkkitehtuurin avulla on mahdollista myös abstrahoida %HUOM
osakkaille (stakeholder) ja muille kehittäjille ohjelmistojärjestelmää helpommin lähestyttäväksi, jolloin kommunikaatio eri osapuolten välillä helpottuu. Abstrahoinnin avulla pystytään yksinkertaistamaan järjestelmän konkreettista toteutusta ja suorittamaan arkkitehtuurista erittelyä (architectural decomposition), jossa muodostetaan epärelevanteista komponenteista mustia laatikoita (black box). Mustien laatikoiden ideana on piilottaa komponenttien sisäistä toteutusta eri tasoilla, jolloin arkkitehtuurista voidaan muodostaa eri abstraktiotason malleja.

Hieman samanlainen jaottelu on määritelty myös Solmsin tutkimuksessa \citep[s. 369]{solms_what_2012}. Siinä ohjelmistoarkkitehtuuri-määritelmä jaetaan kolmeen eri määrittelyluokkaan: korkean tason abstraktioon ohjelmistojärjestelmästä, rakenteita sekä ulkoisia näkyviä ominaisuuksia korostavaan määritelmään ja peruskäsitteistöön sekä rajoitteisiin joiden puitteissa ohjelmistojärjestelmää kehitetään. Näistä kaksi ensimmäistä määritelmää vastaavat samoja kuin \citep[s. 2-7]{gorton_understanding_2011}, mutta tarkentaen kumpaakin omat kuvaustavat. Korkean tason abstraktioita mallinnetaan erilaisten näkymien kautta tai käyttämällä arkkitehtuurallisia malleja, jotka IEEE on määritellyt \cite{ieee_2000}. % IEEE:stä voisi kirjoittaa enemmän    
Rakenteita korostettaessa mallinnuksessa käytetään UML-kaavioita, koska UML määrittelee ohjelmiston arkkitehtuuria osina, joita pystytään rekursiivisesti tarkentamaan haluttaessa. Tarkennuksen avulla osat pystytään kuvaamaan kommunikoivan keskenään erilaisten rajapintojen kautta, osia yhdistäviä suhteita pystytään tarkastelemaan ja erilaisia rajoitteita pystytään luomaan osien välille. Kolmas määritelmä on paljon laajempi näkemys ohjelmistoarkkitehtuurista, koska se kuvaa ohjelmistojärjestelmän peruskäsitteistön ja ominaisuudet siinä ympäristössä, jossa ne ilmeentyvät elementtien, suhteiden ja suunnittelun periaatteiden kautta. Peruskäsitteistö tarjoaa sen käsitteistön, jonka avulla sovelluslogiikka voidaan määritellä. Ominaisuudet liittyvät usein ohjelmistojärjestelmän laadullisiin ominaisuuksiin. Periaatteet voidaan nähdä järjestelmän keskeisinä suunnittelurajoitteina (core design constraints), joiden kokonaisuus muodostaa järjestelmän arkkitehtuurisen tyylin. Ohjelmistojärjestelmän arkkitehtuurin tyyli voi esimerkiksi olla väylät ja suodattimet (pipes and filters). % lisää

Näistä kahdesta eri jaottelusta ohjelmistoarkkitehtuuriin voidaan nähdä merkittävinä yhtäläisyyksinä tarpeen kuvata ohjelmiston rakenteellisuutta, komponenttien välistä kommunikaatiota ja laadullisia vaatimuksia. Kaikki nämä edellämainitut ominaisuudet on saatava  upotettua ohjelmistojärjestelmästä luotavaan arkkitehtuuriseen malliin niin, että ne vastaavat muunmuassa seuraaviin kysymyksiin \citep[s. 31 - 33]{Rozanski:2011:SSA:2072649}: mitkä ovat arkkitehtuurin toiminnalliset elementit; miten nämä elementit kommunikoivat keskenään ja ulkomaailman kanssa; mitä tietoa käsitellään, talletetaan ja esitetään; ja mitä fyysisiä ja ohjelmallisia elementtejä tarvitaan tukemaan näitä  elementtejä. Kuitenkaan arkkitehtuurista luotava malli ei saisi olla monoliittinen malli, joka pyrkii kuvamaan kaiken yhdessä mallissa, koska arkkitehtuuria ei ole mahdollista kuvata vain yhden mallin avulla. Tämä johtuu siitä, että monoliittinen malli on erittäin vaikea ymmärtää, siitä on vaikeaa löytää arkkitehtuurin tärkeimmät ominaisuudet (features) ja se on usein puutteelinen, jäljessä eikä vastaa enää nykyistä ohjelmistojärjestelmää. Ratkaisu tähän on jakaa malli useisiin toisiinsa liittyviin näkymiin (views), jotka esittävät jokainen yhden näkökulman (viewpoint) järjestelmän arkkitehtuuriin keräämällä yhteen toiminnalliset piirteet sekä laadulliset ominaisuudet \citetext{\citealp[s. 33-34]{Rozanski:2011:SSA:2072649}; \citealp[s. 8-9]{gorton_understanding_2011}}. Tämän avulla voidaan tarkastella saavuttaako järjestelmä sille asetetut tavoitteet.


\subsection{Näkymät}

Arkkitehtuuristen näkymien tehtävänä on kuvata ne näkökannat arkkitehtuurista, jotka ovat merkityksellisiä itse asialle, jota näkymä haluaa painottaa \citetext{\citealp{Rozanski:2011:SSA:2072649}; \citealp{may2005survey}}. Yksi tunnetuimmista määritelmistä arkkitehtuuriselle näkymälle on Krutchenin 4+1 näkymämalli (4+1 View Model) \citep[s.7]{gorton_understanding_2011}. Siinä arkkitehtuuri kuvataan neljän näkymän avulla: looginen, prosessi, fyysinen ja kehitys. Looginen näkymä kuvailee esimerkiksi luokkakaavioiden avulla ohjelmistojärjestelmän elementtejä ja niiden välisiä suhteita tarkentaen järjestelmän rakennetta, prosessinäkymä keskittyy ajonaikaisen suorituksen kuvaamiseen tarkentamalla muunmuassa miten samanaikaisuuden hallinta tapahtuu järjestelmässä, fyysinen näkymä keskittyy siihen miten järjestelmän eri komponentit kuvautuvat fyysiselle laitteistolle jossa komponenttia ajetaan ja lopuksi kehitysnäkymä (development view) keskittyy järjestelmän sisäiseen toteutukseen tarkemmalla tasolla kuvailemalla sisäkkäisia pakkauksia tai luokkahierarkiaa. Jokainen näkymä voidaan liittää osaksi toista näkymää skenaarioiden avulla, jotka heijastelevat järjestelmälle asetettuja vaatimuksia. 

Muita näkymämalleja arkkitehtuuriin on useita, joista jokainen kuitenkin tuo esille vähintään jollain tavalla ohjelmistojärjestelmän rakennetta ja yhteyksiä. Tästä esimerkkinä \textit{Views and Beyond} -menetelmä \citep[s.8]{gorton_understanding_2011}. Siinä arkkitehtuurinen malli kuvataan kolmella eri näkymällä: moduuli, joka on järjestelmän rakenteellinen näkymä kuvaten muunmuassa luokkia, pakkauksia, moduulien hajoittamista (decomposition) sekä periytymistä; komponentti ja konnektori, joka kuvailee järjestelmän toiminnallista puolta; allokaatio, joka kuvailee miten prosessit kuvautuvat laitteistotasolla  ja miten ne kommunikoivat keskenään. 

Näiden kahden esitellyn mallin lisäksi löytyy joukko eri tarkoituksiin sopivia näkymämalleja, joilla on omat vahvuudet sekä heikkoudet. Yhteensä viittä eri mallia vertailtiin Mayn tutkimuksessa \citep{may2005survey}, jossa kartoitettiin mahdollisimman laajaa näkymien muodostamaa näkökulmien joukkoa, jotka kattaisivat mahdollisimman laajalti ohjelmistoarkkitehtuurin aluetta (domain). Vertailtavat arkkitehtuurit olivat 4+1-malli, SEI:n (Software Engineering Institute) \textit{Views and Beyond}, ISO:n (International Organization for Standardization) \textit{Reference Model of Open Distributed Processing}, Siemens'n \textit{Four View Model} ja \textit{Rational Architecture Description Specification}. 
	
	Näkökulmat (viewpoints) liittyvät vahvasti arkkitehtuurisin näkymiin




%onko testatavuus missäkin tapauksessa eri näkymien yli menevä vai voiko tai kannattaako sille jossain tilanteessa olla ihan oma näkymänsä, ja jos niin millainen se sitten olisi

\subsection{Arkkitehtuuri osana ohjelmistonkehitystä}


\subsection{Vaatimukset}
Vaatimusten keräys vs ei-toiminnalliset vs toiminnalliset.



\subsection{Historia}


\section{Ohjelmistojen laadulliset tekijät}
\subsection{Arviointi}
\subsection{Tradeoffit}

\section{Testattavuus laadullisena tekijänä}
\subsection{Testattavuuden merkitys}

\section{Arkkitehtuurin vaikutus testattavuuteen}
\subsection{Jotain}


\section{Yhteenveto}


Yleensä hieman johdantoa lyhyempi.

Muistuttaa mieleen tutkimuskysymyksen, mainitsee tärkeimmät tulokset ja niiden perusteet. Keskittyy impaktiin ja esimerkiksi suosituksiin. Ei vain summeeraa luku kerrallaan aikaisempaa tekstiä.



% --- References ---
%
% bibtex is used to generate the bibliography. The babplain style
% will generate numeric references (e.g. [1]) appropriate for theoretical
% computer science. If you need alphanumeric references (e.g [Tur90]), use
%
% \bibliographystyle{babalpha-lf}
%
% instead.

\bibliographystyle{babalpha-lf}
%\bibliographystyle{apsr}
\bibliography{references-fi}


% --- Appendices ---

% uncomment the following

% \newpage
% \appendix
% 
% \section{Esimerkkiliite}

\end{document}
