% --- Template for thesis / report with tktltiki2 class ---
% 
% last updated 2013/02/15 for tkltiki2 v1.02

\documentclass[finnish]{tktltiki2}

% tktltiki2 automatically loads babel, so you can simply
% give the language parameter (e.g. finnish, swedish, english, british) as
% a parameter for the class: \documentclass[finnish]{tktltiki2}.
% The information on title and abstract is generated automatically depending on
% the language, see below if you need to change any of these manually.
% 
% Class options:
% - grading                 -- Print labels for grading information on the front page.
% - disablelastpagecounter  -- Disables the automatic generation of page number information
%                              in the abstract. See also \numberofpagesinformation{} command below.
%
% The class also respects the following options of article class:
%   10pt, 11pt, 12pt, final, draft, oneside, twoside,
%   openright, openany, onecolumn, twocolumn, leqno, fleqn
%
% The default font size is 11pt. The paper size used is A4, other sizes are not supported.
%
% rubber: module pdftex

% --- General packages ---

\usepackage[utf8]{inputenc}
\usepackage[T1]{fontenc}
\usepackage{lmodern}
\usepackage{microtype}
\usepackage{amsfonts,amsmath,amssymb,amsthm,booktabs,color,enumitem,graphicx}
\usepackage[pdftex,hidelinks]{hyperref}
\usepackage{microtype}

\DisableLigatures[?,!]{encoding=T1}

% Automatically set the PDF metadata fields
\makeatletter
\AtBeginDocument{\hypersetup{pdftitle = {\@title}, pdfauthor = {\@author}}}
\makeatother

% --- Language-related settings ---
%
% these should be modified according to your language

% babelbib for non-english bibliography using bibtex
\usepackage[fixlanguage]{babelbib}
\selectbiblanguage{finnish}

% add bibliography to the table of contents
\usepackage[nottoc]{tocbibind}
% tocbibind renames the bibliography, use the following to change it back
\settocbibname{Lähteet}

% --- Theorem environment definitions ---

\newtheorem{lau}{Lause}
\newtheorem{lem}[lau]{Lemma}
\newtheorem{kor}[lau]{Korollaari}

\theoremstyle{definition}
\newtheorem{maar}[lau]{Määritelmä}
\newtheorem{ong}{Ongelma}
\newtheorem{alg}[lau]{Algoritmi}
\newtheorem{esim}[lau]{Esimerkki}

\theoremstyle{remark}
\newtheorem*{huom}{Huomautus}


% --- tktltiki2 options ---
%
% The following commands define the information used to generate title and
% abstract pages. The following entries should be always specified:

\title{The Functional Architecture Modeling Method applied on Web Browsers}
\author{Kristian Wahlroos - 014417003}
\date{\today}
\level{Referaatti}

% The following can be used to specify keywords and classification of the paper:


% classification according to ACM Computing Classification System (http://www.acm.org/about/class/)
% This is probably mostly relevant for computer scientists
% uncomment the following; contents of \classification will be printed under the abstract with a title
% "ACM Computing Classification System (CCS):"
% \classification{}

% If the automatic page number counting is not working as desired in your case,
% uncomment the following to manually set the number of pages displayed in the abstract page:
%
% \numberofpagesinformation{16 sivua + 10 sivua liitteissä}
%
% If you are not a computer scientist, you will want to uncomment the following by hand and specify
% your department, faculty and subject by hand:
%
% \faculty{Matemaattis-luonnontieteellinen}
% \department{Tietojenkäsittelytieteen laitos}
% \subject{Tietojenkäsittelytiede}
%
% If you are not from the University of Helsinki, then you will most likely want to set these also:
%
% \university{Helsingin Yliopisto}
% \universitylong{HELSINGIN YLIOPISTO --- HELSINGFORS UNIVERSITET --- UNIVERSITY OF HELSINKI} % displayed on the top of the abstract page
% \city{Helsinki}
%


\begin{document}

% --- Front matter ---

\frontmatter      % roman page numbering for front matter

\maketitle        % title page

% \tableofcontents  % table of contents

% --- Main matter ---

\mainmatter       % clear page, start arabic page numbering

% Write some science here.

Wilbert Seelen, Shaheen Syedin ja Sjaak Brinkkemperin julkaisemassa tutkimuksessa ``The Functional Architecture Modeling Method applied on Web Browsers`` vuodelta 2014 ottaa analysoitavaksi funktionaalisen arkkitehtuurin mallinnuksen (Functional Architecture Modelling), jonka pohja luotiin Sjaak Brinkkemperin ja Stella Pachidin julkaisussa ``Functional Architecture Modeling for the Software Product Industry`` vuodelta 2010. Tutkimuksessa myös kerrotaan menetelmä luoda funktionaalinen malli järjestelmästä. Itse analysointi julkaisussa tapahtuu tapaustutkimuksen kautta, jossa analysoidaan mallin toimivuutta luomalla funktionaalinen malli kuvitteellisesta web-selaimesta, johon on lisätty sama toiminnallisuus kuin oikean maailman selaimessa olisi. Tätä mallia verrataan toiseen tutkimukseen, jossa koodianalyysi-työkalulla luotiin korkean tason malli oikean maailman web-selaimen lähdekoodista. Funktionaalinen malli koettiin onnistuneeksi, koska mallit vastasivat loppuanalyysissä toisiaan.\\

Tutkimus ``The Functional Architecture Modeling Method applied on Web Browsers`` alkaa kertomalla, että arkkitehtuurin design ja dokumentointi on projekteissa usein keskeneräistä tai puuttuu jopa kokonaan. Tähän ongelmaan he esittävät ratkaisuksi funktionaalisen arkkitehtuurin mallinnuksen. He väittävät, että tämä arkkitehtuurin mallinnustapa on virtaviivaistettu tyyli, joka antaa tilaa nopeille arkkitehtuurin iteraatiolle, koska keskittyminen on ollut mallin helppokäyttöisyydessä ja kommunikaatioon tärkeydessä. Tämä uusi malli myös yhdistää funktionaalista ja teknistä arkkitehtuuria ja antaa visualisaation järjestelmän ydin toiminnallisuuteen ja sen avulla nähdään järjestelmän eri osien lopulliset tarkoitukset.  

Tutkimuksen tarkoituksena on ollut pienentää dokumentoinnin ja lopullisen toiminnallisuuden välistä kuilua tämän uuden mallin avulla. He jatkavat kertomalla, että itse mallilla on kaksi päätarkoitusta: mahdollistaa selkeä korkean tason näkymä projektiin sekä ylläpitää hyviä käytänneitä ohjelmoinnissa, koska se pakottaa jakamaan toteutettavan järjestelmän toiminnallisiin moduuleihin jo projektin alkuvaiheessa. He määrittelevät seuraavaksi vastattavan tutkimuskysymyksen muodossa ``Miten toiminnallisuutta voidaan ilmaista ohjelmistoarkkitehtuurien malleissa?``, johon he vastaavat, että adaptoimalla Brinkkeemperin ja Pachidin luomaa metodia. 

Nykyisestä kirjallisuudesta he löytävät materiaalia tutkimuksen tueksi funktionaalisen arkkitehtuurin mallinnuksesta ja vaatimuksista ohjelmistoarkkitehtuurin mallinnuksessa.
\iffalse
Funktionaalisen arkkitehtuurin mallinuksesta he viittaavat Brinkkemperin tutkimukseen, jossa tutkittiin funktionaalisen arkkitehtuurin diagrammien käyttöä ja jonka tulos oli, että niiden käyttö on rajoittunutta tiettyihin alueisiin ja usein rakenteiltaan sekavahkoa. He jatkavat viittaamalla myös Vlietin tutkimukseen, jossa tultiin päätelmään, että arkkitehtuurin suunnittelun pitäisi sijaita vaatimusten keräysten ja teknisen suunnittelun välissä. 
Vaatimuksista 
\fi
He ottavat esille ADL-mallien (Architecture Description Language) ongelmia; ADL yrittää mallintaa koko järjestelmän arkkitehtuuria laajamittaisesti, ADL ei täytä suunnittelijoiden vaatimuksia, vaikeasti ymmärrettävä sekä ADL tukee vain yhtä näkymää. He nostavatkin esille UML-mallin, josta on tullut nykypäivänä niin sanottu ``de facto`` -standardi arkkitehtuurin kuvauksessa. Syy tähän on, että UML on kevyempi käyttää ja visuaalisen muotonsa takia helpommin lähestyttävä. 

Seuraavaksi tutkimuksessa avataan itse menetelmä arkkitehtuurin kuvaukseen, joka on muokattu hieman aiemmasta ``Functional Architecture Modeling for the Software Product Industry``-julkaisusta. Uudessa menetelmässä suunnittelu ei enää keskity moduläärisyyteen, vaihtelevuuteen tai yhteentoimivuuteen, vaan on haluttu korostaa helppokäyttöisyyttä, kommunikoinnin tärkeyttä, tukea useille näkymille, rajoittavuuden vähentämistä sekä nykyisiä laatuvaatimuksia malleissa. Itse luotu malli koostuu neljästä eri vaiheesta: vaatimusten keräys, ominaisuuksien mallinnus (feature modelling), toiminnallisen arkkitehtuurin suunnitteleminen ja lopuksi teknisen arkkitehtuurin suunnitteleminen. Vaatimusten keräyksessä kerättään toiminnalliset ja tekniset vaatimuksen osakkailta, ominaisuuksien mallinnuksessa vaatimuksen yhdistetään aiheiden mukaan, toiminnallisen arkkitehtuurin suunnittelemisessa edellisen vaiheen tuotokset yhdistetään erillisiksi moduuleiksi, joiden välille määritellään yhteydet ja niiden tyypit, toiminnallisen arkkitehtuurin suunnittelemisessa moduuleita tarkennetaan teknisten vaatimusten avulla. Lopullinen dokumentaatio, joka on käytännössä UML-tyylinen malli, näytetään osakkaille. Tätä mallia pystytään näin tarkentamaan iteraatioittain sekä ohjelmointi pystytään jo aloittamaan. 

He analysoivat metodin toimintaa tapaustutkimuksen kautta, jossa mallinnettiin web-selainta ja jonka vaatimuksista mukaan otettiin vain ne vaatimukset, jotka ovat ydintoiminnallisuutta web-selaimelle. Tapaustutkimuksessa luotiin ensin korkean tason näkymä, josta jokaista luotua moduulia sitten tarkennettiin tarpeen mukaisesti. Yhteensä kolme eri mallia luotiin tutkimuksessa: korkean tason toiminnallinen arkkitehtuuri, tarkennettu kuvaus eräästä moduulista sekä korkean tason tekninen arkkitehtuuri. Analysointivaiheessa tutkittiin, että miten tekninen arkkitehtuuri vastaa koodianalyysin avulla tuotettua arkkitehtuuria. Tuloksena saatiin, että ainoa erilaisuus löytyi vain yhdestä moduulista, joka oli ylimääräinen. Syy tälle löytyi heidän mukaansa siitä, että moduuli on todennäköisesti oikeissa selaimisssa sulautettu muihin moduuleihin. 

Lopullisissa päätelmissä he tulivat tulokseen, että nykyiset  
 







% --- References ---
%
% bibtex is used to generate the bibliography. The babplain style
% will generate numeric references (e.g. [1]) appropriate for theoretical
% computer science. If you need alphanumeric references (e.g [Tur90]), use
%
% \bibliographystyle{babalpha-lf}
%
% instead.

%\bibliographystyle{babalpha-lf}
%\bibliography{references-fi}


% --- Appendices ---

% uncomment the following

% \newpage
% \appendix
% 
% \section{Esimerkkiliite}

\end{document}
